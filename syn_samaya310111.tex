\documentclass{emulateapj}
%\documentclass[12pt,preprint]{aastex}
\usepackage{graphicx}
\usepackage{bm}
\usepackage{epsfig}
%\documentclass{emulateapj}
%\usepackage{epsfig,apjfonts}
\bibliographystyle{apj}

\usepackage{color}
\newcommand{\samaya}[1]{\textcolor{blue}{\bf #1}}
\newcommand{\daniel}[1]{\textcolor{green}{\bf #1}}
\newcommand{\mycomment}[1]{\textcolor{red}{\bf[ #1]}}
\newcommand{\scott}[1]{\textcolor{cyan}{#1}}

%\usepackage{epsfig,apjfonts}

\def\showprivate{
   \global\long\def\private{
       \typeout{**INCLUDING PRIVATE MATERIAL**}
   }
}

\def\hideprivate{
   \global\long\def\private##1{
       %\typeout{**EXCLUDING PRIVATE MATERIAL**}
       \iffalse ##1 \fi
   }
}

\hideprivate    %% This is the default
\shorttitle{Compact binary inspiral and short gamma-ray bursts}
\shortauthors{Nissanke et al.}

\begin{document}

\title{ Sky localisation for inspiralling compact binaries using a
  network of ground based gravitational wave interferometers}

\author{Samaya Nissanke\altaffilmark{1,2,3} et al.}

\altaffiltext{1}{Caltech, Theoretical Astrophysics, California Institute of Technology, Pasadena, California 91125}

\altaffiltext{2}{JPL, 4800 Oak Grove Drive, Pasadena, California 91109}

\altaffiltext{3}{CITA, University of Toronto, 60 St.~George St.,
  Toronto, ON, M5S 3H8, Canada}

\altaffiltext{4}{Theoretical Division, Los Alamos National Laboratory,
  Los Alamos, NM 87545}

\altaffiltext{5}{Department of Astronomy and Astrophysics, University
  of Chicago, Chicago, IL 60637}

\altaffiltext{6}{Dept.\ of Physics and MIT Kavli Institute, 77
  Massachusetts Avenue, Cambridge, MA 02139}


\begin{abstract}
\samaya{Era of multimessenger or transpectral astronomy is nearing upon us. Advent of first kHZ GW detection with advanced LIGO. Most promising sources are inspiralling and merging compact binaries. Also happens to be the main progenitor model for short hard bursts (but not certain). GW measurements will allow us to discriminate between different central engine models, constrain beaming and provide sky localisation. Will be of immense use to EM community especially large-scale optical surveys with increasing sensitivity,
rapid cadence, and sky-area coverage--all will lead to a revolution in
the study of transient objects. GW events will either provide a
trigger to GRB observatories or LSST like transient surveys, or at
least provide an insight into the distribution of inspiralling NS/NS
and NS/BHs. We study the impact that gravitational wave measurements of
inspiralling compact binaris will have using different proposed GW detectors.}
\end{abstract}

\keywords{cosmology: distance scale---cosmology: theory---gamma rays:
bursts---gravitational waves}

\section{Introduction}

\samaya{Maximise science potential from future GW detection. Must use
  both GW and EM channels. Frutiful exchange between two communities
  necessary. Earlier work (standard siren) considered how EM
  counterparts will aid GW measurements and constrain cosmological
  parameters. This work considers how solely using measurements from
  GW channel will aid understanding and observation strategy of their
  EM counterparts. In particular, sky localisation, central engine and
  extent of collimation looking at the case where we know or do not
  know about the time of merger. Promised observations of gravitational waves and high energy
neutrinos, together with complementary electromagnetic waves, herald a new era of multimessenger or transpectral astronomy.}

With promised observations from gravitational wave (GW) and high
energy neutrino detectors in the next decade, an unprecedented era of transpectral or multimessenger
astronomy is nearing upon us. In order to
maximise their full scientific potential, accurate sky
localisation of gravitational wave events are necessary for the
follow up studies using
traditional electromagnetic (EM) telescopes. Inspiralling and merging
compact binaries (binary systems involving neutron stars (NSs) and/or
stellar mass black holes(BHs)) are
amongst the most numerous sources for future gravitational wave
detectors. A subset of such systems, NS-NS and NS-BH (stellar mass) binaries, are considered to be the most likely progenitor model for
the majority of short hard burst (henceforth, denoted SHB) observations. Joint GW and EM
observations will undisputedly give us complementary information about
the nature and underlying physics of the central engine and outflows of such bursts.

Our work is founded on the methodology and techniques developed in previous work
by the authors of NHHDS09 and NHHDS10.  NHHDS09 and NHHDS10 consider standard sirens, the GW counterpart to traditional
standard candles, using networks comprising advanced ground-based GW
detectors. NHHDS09 assumes that an SHB observation either has been known
earlier or is coincident to that of a GW measurement of an inspiralling
NS-NS or NS-BH binary. In this instance, the SHB observation aids the GW
measurement three ways by indicating: (i) a merger time for the
system, (ii) the sky localisation and thus, reducing the
dimensionality of the source parameter space, and (iii) the
redshift, the critical quantity when mapping out the luminosity distance-redshift relationship.

In contrast, this work examines how the use of {\sl only} GW
measurements of such sources provides their sky localisation, thereby contributing to
any proposed observation strategy of their EM counterparts. This is of
particular interest to future GRB satellites, and current and upcoming large-scale optical surveys with high sensitivity,
rapid cadence, and large sky-area coverage. Several recent works have
explored sky localisation errors in the context of burst (unmodelled waveform) sources
for advanced GW detector networks, relying on analytically derived
timing formulae. This is the first work that {\sl explicitly} computes
sky localisation measurement errors for non-spinning NS-NS and NS-BH populations using Markov Chain Monte Carlo
(MCMC) techniques for advanced GW detector networks. In the context of
initial Ligo detectors and the LSC, works such as Van der Sluys {\sl
  et al.}, are instrumental in addressing such concerns using MCMC
techniques for spin-precessing NS-BH systems. \samaya{Most signals for
  Ligo are detected at threshold, therefore necessitating MCMC or
  similar Bayesian approach. We will only look at NS-NS systems here,
  results for NS-BH are similar in the main features, only difference
  is with detectable volume.}



\section{GWs from Inspiralling Compact Binaries}


Gravitational waveforms rely on successfully modelling the local
dynamics of the source and the subsequent gravitational wave emission
detectable by an observer at infinity. In general, waveforms for
GW driven compact binaries
comprise three phases: the so-called `inspiral', `merger' and
`ringdown'. \samaya{Cut the below}. Post-Newtonian (PN) techniques (an expansion in gravitational
potential $M/r$, or equivalently for bound systems, orbital speed
$v^2$) provide an extremely accurate description of the gradual
adiabatic {\it inspiral}, when each compact body slowly spirals together due
to the radiative loss of orbital energy and angular momentum through
gravitational waves. Eventually, the two bodies become sufficiently
close that the weak field perturbative PN expansion is no
longer valid. Modelling this brief {\sl merger} phase requires numerical
simulations that directly solve the full Einstein
field equations. Finally, the system experiences the {\sl ringdown}
phase, where it settles down
to a weakly perturbed, single black hole. 

In this work, we use only the inspiral portion of the waveform where
the majority of the SNR for NS-NS and NS-BH binaries are
accumulated in the band of
advanced detectors (for instance, NS-NS binaries will
radiate approximately $10^4$ cycles in the band of advanced LIGO, and NS-BH
binaries $10^3$ cycles). In particular, we apply the non-spinning restricted 2PN
waveform in the frequency domain, such that the two gravitational wave
polarisations are:
\begin{eqnarray}
\tilde{h}_+(f) &=& \sqrt{\frac{5}{96}}\frac{\pi^{-2/3} {\mathcal
M}_z^{5/6}}{D_L}{\cal A}_+ f^{-7/6} e^{i\Psi(f)} \, ,
\label{eq:freqdomainhp}
\\
\tilde{h}_\times(f) &=& \sqrt{\frac{5}{96}}\frac{\pi^{-2/3} {\mathcal
M}_z^{5/6}}{D_L}{\cal A}_\times f^{-7/6} e^{i\Psi(f) - i\pi/2} \, ,
\label{eq:freqdomainhc}
\end{eqnarray}
where $ {\mathcal
M}_z = (1+z) m_1^{3/5} m_2^{3/5}/(m_1 + m_2)^{1/5}$ is the redshifted chirp mass of the system for two masses $m_1$ and $m_2$, $f$ is the GW frequency which evolves with a characteristic chirp given by Eq. (10) in NHHDS09 and $D_L$ is the luminosity
distance to the binary. The functions ${\cal A}_{+,\times}$ depend on the binary's orientation (Eqs. (7) and (8) in NHHDS09). The phase function $\Psi(f)$ in Eqs.\
(\ref{eq:freqdomainhp}) and (\ref{eq:freqdomainhc}) is given at 2PN order by Eq. (14) in NHHDS09 and is a function of ${\mathcal
M}_z$, the redshifted reduced mass $\mu_z$, the parameters $t_c$ and $\Phi_c$, the time and orbital phase when the frequency $f$ diverges in the PN approximation.  
We use units with $G = 1 = c$; conversion factors are $M_\odot
\equiv GM_\odot/c^2 = 1.47\,{\rm km}$, and $M_\odot \equiv
GM_\odot/c^3 = 4.92 \times 10^{-6}\,{\rm seconds}$. 

The predicted inspiral waveform encodes source parameters such as the binary's
sky position,  luminosity distance, its `redshifted' masses, its
orientation in the sky, the time of merger and other parameters
intrinsic to the source. Critically, the inspiral does not contain any information regarding the source's redshift.  

\samaya{Sentence on timing which is main effect.}  
\section{Methodology}
\samaya{Standard stuff: Basic review of parameter estimation, MCMC methods, Priors, Selection of Binaries}

This section summarises the methodology used and detailed in NHHDS10. For conciseness, we refer the reader to Sections 3 and 4 of NHHDS09.  

When estimating source parameters, the central quantity of interest is the posterior probability
density function:
\begin{equation}
\label{eq:postPDF}
p({\boldsymbol \theta} \, | \, {\bf s}) = {\cal N} \, p^{(0)}
({\boldsymbol{\theta}}) {\cal L}_{\rm TOT} ({\bf s} \, |
\,{\boldsymbol{\theta}}) \,, 
\end{equation}
where ${\boldsymbol{\theta}} = \{ {\cal M}_z, \mu_z, D_L, t_c, \cos \theta, \phi, \cos \iota, \psi,
\Phi_c \}$ is the vector set of parameters. As is the standard practice,
we assume here that a GW has been detected. In Eq.\
(\ref{eq:postPDF}), ${\cal N}$ denotes a normalization constant,
$p^{(0)}({\boldsymbol{\theta}})$ is the PDF that represents the prior
probability that a measured GW is described by the parameters
$\boldsymbol{\theta}$, and ${\cal L}_{\rm TOT} (\bf{s} \, | \,
{\boldsymbol \theta} )$ is the total {\it likelihood function} (e.g.,
\citealt{mackay03}).  The likelihood function measures the relative
conditional probability of observing a particular set of data $\bf{s}$
given a measured signal ${\bf h}$ depending on some unknown set
of parameters, $\boldsymbol{\theta}$ and an instant of noise, ${\bf n}$.  Because we
assume that the noise is independent and uncorrelated at each detector
site, the total likelihood function is the product of
the individual likelihood at each detector:
\begin{equation}
\label{eq:totLike}
{\cal L}_{\rm TOT} ({\bf s} \, | \,{\boldsymbol \theta}) = \Pi_{a}
{\cal L}_a (s_a \, | \,{\boldsymbol \theta})\;,
\end{equation}
where ${\cal L}_a$, the likelihood computed at detector $a$, is given
by \citep{finn92}
\begin{equation}
\label{eq:Like}
{\cal L}_a \, (s \, | \,{\boldsymbol \theta}) = \, e^{ -
\big( h_a({\boldsymbol \theta}) - s_a \, \big| \, h_a({\boldsymbol
\theta}) - s_a \big)/2 } \, .
\end{equation}
The inner product $\left( \ldots | \ldots \right)$ on the vector space
of signals is defined as
\begin{equation}
(g|h) = 2 \int_0^{\infty} df \frac{\tilde{g}^*(f)\tilde{h}(f) +
\tilde{g}(f)\tilde{h}^*(f)}{S_n(f)} \, .
\label{eq:innerproduct}
\end{equation}
where $S_n(f)$ is the one-sided power density spectrum.

For completeness, we now provide some important definitions. The {\it true}\/ SNR at detector $a$, associated with a given instance of
noise for a measurement at a particular detector, is defined as:
\begin{eqnarray}
\left({S\over N}\right)_{a, {\rm true}} & = & { \left( h_a \, |
\, s_a \right) \over \sqrt{ \left( h_a \, | \, h_a\right) } }\;.
\label{eq:snr_true}
\end{eqnarray}
For an ensemble of detector noise realisations $n_a$, the {\it
average} SNR at detector $a$ is given by
\begin{equation}
\label{eq:snr_ave}
\left({S\over N}\right)_{a, {\rm ave}} = {{(h_a | h_a)}\over
{ {\rm rms}\ (h_a|n_a)}} = (h_a|h_a)^{1/2}.
\end{equation}
Subsequently, the {\it average} SNR of a coherent network of detectors
is:
\begin{eqnarray}
\left({S\over N}\right)_{{\rm ave}} & = & \sqrt{\sum_a \left({S\over
N}\right)^2_{a, {\rm ave}}}\ \ .
\label{eq:snr_tot_ave}
\end{eqnarray}

As detailed in Section 3.3 of NHHDS09, we use Markov Chain Monte Carlo (MCMC) methods to explore
the full posterior PDF, given by Eq.\ (\ref{eq:postPDF}). The MCMC algorithm used is based on a generic version of
CosmoMC, described in \cite{lewis02}.\footnote{See http://cosmologist.info/cosmomc/}
The choice of priors is important in our MCMC approach, especially for
signals detected at the threshold SNR.  We take the prior
distributions in chirp mass ${\cal M}_z$, reduced mass $\mu_z$,
polarization angle $\psi$, coalescence time $t_c$, luminosity distance
$D_L$, and coalescence
phase $\Phi_c$ to be {\it flat} over the region of sample space where
the binary is detectable at a SNR = 3.5. For our sample with isotropic inclination
distribution, we use $p^{(0)}(\cos \iota) = {\rm constant}$ over the
range $[-1,1]$. 

As our previous work indicates, how we select our binaries is central
to the derivation of representative sky localisation statistics for binary
populations. We simulate a million binaries out to a $z = 1$,
assuming a constant comoving volume density  (\citealt{peebles93},
\citealt{hogg99}) in a $\Lambda$CDM Universe with
$H_0=70.5\ \mbox{km}/\mbox{sec}/\mbox{Mpc}$, $\Omega_{\Lambda}=0.726$,
and $\Omega_{m}=0.2732$ \citep{komatsu09} with random sky positions and orientations. We
consider three cases where each binary in the total population is selected: (i). by triggering on
the {\sl average} total network SNR threshold of 8.5 (see figure \ref{fig:detected_binaries_caseI}), (ii). by requiring that both Ligo
Hanford and Ligo Livingston each has an {\sl average} SNR of 6 or
higher and an {\sl average} total network SNR threshold above 12 (see figure \ref{fig:detected_binaries_caseII}). Case I uses a similar detection
criteria as {\sl network} thresholding criteria developed in
NHHDS09 and NHHDS10. However, the value of our network detection
threshold used here is
$8.5$, which is in contrast to the value of $7.5$, used in NHHDS09 and NHHDS10.
The earlier works assume that an EM counterpart has been observed {\sl
  prior} or {\sl coincident} with the GW measurement, significantly
reducing the number of templates to be searched over, and hence,
lowering the detection threshold.

\begin{figure}[t]
\centering 
\includegraphics[width=0.9\columnwidth]{Net8.5.eps}
\caption{Detected NS-NS binaries for our various detector networks as
a function of sky position $(\cos\theta,\phi)$ using an average
network threshold SNR of 8.5.  The lower right panel
shows the binaries detected by a five-detector network (both LIGO
sites, Virgo, AIGO, and LCGT).  We find that LIGO plus Virgo only detects $762$ events;
LIGO, Virgo, and AIGO detect $1142$ events; LIGO, Virgo,
and LCGT, detect $1076$ events; and  LIGO, Virgo, AIGO
and LCGT, detect $1500$ events.  Detections are more
uniformly distributed on the sky in networks that include LCGT; AIGO
improves coverage in two of the sky's quadrants.  Our coordinate
$\phi$ is related to right ascension $\alpha$ by $\phi=\alpha-$GMST,
where GMST is Greenwich Mean Sidereal Time; $\theta$ is related to
declination $\delta$ by $\theta = \pi/2 - \delta$.}
\label{fig:detected_binaries_caseI}
\end{figure} 

\begin{figure}[t]
\centering 
\includegraphics[width=0.9\columnwidth]{Ligo6Net12.eps}
\caption{Detected NS-NS binaries for our various detector networks as
a function of sky position $(\cos\theta,\phi)$ in the Case II
detection threshold scenario. Here, we require both that each LIGO detector
detects the event with a SNR of at least 6, and that the {\sl average}
network threshold is at least 12. We find that LIGO plus Virgo only detects $208$ events;
LIGO, Virgo, and AIGO detect $275$ events; LIGO, Virgo,
and LCGT, detect $240$ events; and  LIGO, Virgo, AIGO
and LCGT, detect $293$ events.  }
\label{fig:detected_binaries_caseII}
\end{figure}

\section{Results}

This section presents our results.  

\subsection{Individual Binaries}



\begin{figure}[t]
\centering 
\includegraphics[width=0.98\columnwidth]{highSNR.eps}
\includegraphics[width=0.98\columnwidth]{binary2.eps}
\includegraphics[width=0.98\columnwidth]{binary3.eps}
\includegraphics[width=0.98\columnwidth]{binary4.eps}
\caption{PLOTS NEED TO BE REDONE: Sky error
  ellipses for four binaries under different network
  configurations. The origin (0,0) of each plot represents the
  source's true positions.}
\label{fig:highSNRellipse}
\end{figure} 

\subsection{Case I: Triggering on a network of detectors}
\samaya{How well do various plausible networks determine the sky
  localisation to different binaries? With or without knowledge of the
time of merger?}

\begin{figure}[h!]
\centering 
\resizebox{\hsize}{!}{\includegraphics[angle=90,,width=0.95\textwidth]{normalized_cumulative_distributions_1sigma.eps}}
\resizebox{\hsize}{!}{\includegraphics[angle=90,,width=0.95\textwidth]{normalized_cumulative_distributions_2sigma.eps}}
\resizebox{\hsize}{!}{\includegraphics[angle=90,,width=0.95\textwidth]{normalized_cumulative_distributions_3sigma.eps}}
%\includegraphics[width=0.98\columnwidth]{normalized_cumulative_distributions_1sigma.eps}
\caption{Normalized cumulative distributions for a sample of
  binaries (29 binaries randomly selected from the total detected
  sample) in Case I detection scenario. Black line denotes
  Ligo+Virgo+Aigo+LCGT network, green line represents Ligo+Virgo+Aigo,
  red line is Ligo+Virgo+LCGT, blue line indicates Ligo+Virgo only.  
}
\label{fig:Normcd}
\end{figure} 

\begin{figure}[h!]
\centering 
\resizebox{\hsize}{!}{\includegraphics[angle=90,,width=0.95\textwidth]{cumulative_distributions_1sigma.eps}}
\resizebox{\hsize}{!}{\includegraphics[angle=90,,width=0.95\textwidth]{cumulative_distributions_2sigma.eps}}
\resizebox{\hsize}{!}{\includegraphics[angle=90,,width=0.95\textwidth]{cumulative_distributions_3sigma.eps}}
%\includegraphics[width=0.98\columnwidth]{normalized_cumulative_distributions_1sigma.eps}
\caption{Cumulative distributions for a sample of
  binaries (29 binaries randomly selected from the total detected
  sample) in Case I detection scenario. Black line denotes
  Ligo+Virgo+Aigo+LCGT network, green line represents Ligo+Virgo+Aigo,
  red line is Ligo+Virgo+LCGT, blue line indicates Ligo+Virgo only.  }
\label{fig:cd}
\end{figure} 



\subsection{Case II: Triggering on single detectors}








\section{Acknowledgements}
Sathya, Sterl, Curt Cutler, Michele Vallisneri, Derek Fox, Shinichiro
Ando, Kip Thorne, Alan Weinstein, Mansi Kaliswal, Eran Ofek, JM Desert,
CITA Sunnyvale cluster.



\bibliography{sirens}
\end{document}
